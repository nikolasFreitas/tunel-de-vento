\documentclass[a4paper]{article}

%% Language and font encodings
\usepackage[brazil]{babel}
\usepackage[utf8x]{inputenc}
\usepackage[T1]{fontenc}

%% Sets page size and margins
\usepackage[a4paper,top=3cm,bottom=2cm,left=3cm,right=3cm,marginparwidth=1.75cm]{geometry}

%% Useful packages
\usepackage{amsmath}
\usepackage{graphicx}
\usepackage[colorinlistoftodos]{todonotes}
\usepackage[colorlinks=true, allcolors=blue]{hyperref}
\usepackage{ragged2e}
\usepackage{setspace}
\usepackage{enumitem}
\usepackage{float}
\usepackage{caption}
% \usepackage{biblatex}

\newcommand*{\Scale}[2][4]{\scalebox{#1}{$#2$}}%
\newcommand*{\Resize}[2]{\resizebox{#1}{!}{$#2$}}%

\title{Tubo de Pitot no túnel de vento}
\author{Nikolas Bernardes Vieira de Freitas}

\setlength{\parindent}{2em}
\renewcommand{\baselinestretch}{1.5}

\begin{document}
\maketitle

\section{Resumo}


\section{Introdução}

\section{Embasamento teórico}

\subsection*{Equações utilizadas:}
    {

    }

\section{Materiais}
    \begin{enumerate}[label=(\roman*)]
        \item Paquímetro \textit{ (incerteza: $1 * 10^{-3}$cm) }.
    \end{enumerate}

\section{Montagem}


\section{Procedimento}

\subsection{Condução do experimento}

 \subsection{Considerações sobre o procedimento}
  \begin{enumerate}
      \item
  \end{enumerate}


\section{Dados obtidos}

\subsection{Tabela de dados}
\begin{center}
    \begin{tabular}{ |c|c| }
        \hline
        \multicolumn{2}{|c|}{Quantidade de lançamentos} \\

        \hline
        \textbf{Lançamento} & \textbf{Tempo(s)} \\ [0.5ex]
        \hline
        1 & 0,448 \\
        \hline
        2 & 0,378 \\
        \hline
        3 & 0,378 \\
        \hline
        4 & 0,432 \\
        \hline
        5 & 0,378 \\
        \hline
        6 & 0,377 \\
        \hline
        7 & 0,432 \\
        \hline
        8 & 0,434 \\
        \hline
        9 & 0,465 \\
        \hline
        10 & 0,377 \\
        \hline
    \end{tabular}
\end{center}

\subsection{Resultado dos dados obtidos}
\begin{enumerate}
    \item Lançamentos: $ 10 $
    \item Média da aceleração da gravidade: $ 9,53 m/s^2 $
    \item Desvio padrão da média da aceleração da gravidade: $ 1,60 m/s^2 $
    \item incerteza da aceleração da gravidade: $ 0,50 m/s^2 $
\end{enumerate}

\section{Discussão dos dados}
    Segue as pressões calculadas para cada diâmetro do túnel de vento.
        \begin{center}
              \includegraphics[width=.6\linewidth]{img/graph.jpg}
              \captionof{figure}{Histograma da aceleração da gravidade ($ m/s^2 $)}
              \label{graph}
        \end{center}

\section{Conclusão}
    coisas aqui

\begin{thebibliography}{1}

\bibitem{knuthwebsite}
    Wikipedia: Lei da queda dos corpos,
    \\\texttt{https://pt.wikipedia.org/wiki/Lei\_da\_queda\_dos\_corpos}
\end{thebibliography}

\end{document}
